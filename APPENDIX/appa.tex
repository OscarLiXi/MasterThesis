\chapter{Implementation for $\Psi$ of BCAGM+LIN}
\label{chap:appendix}
% \begin{table}
% \caption{Armadillos}
% \label{arm:table}
% \begin{center}
% \begin{tabular}{||l|l||}\hline
% Armadillos & are \\\hline
% our	   & friends \\\hline
% \end{tabular}
% \end{center}
% \end{table}
\begin{algorithm}
    \caption{$\Psi(\bA, \bx^0)$} 
    \KwIn {
	$\bA \in \RR^{n\times n}$, $\bx^0 \in \RR^n$ as starting point
    }
    \KwOut {
	$\bx^* \in \RR^n$ s.t. $\inner{\bx^*, \bA\bx^*} \geq \inner{\bx^0, \bA\bx^0}$
    }
    \textbf{Repeat} 
    \begin{enumerate}
	\item $\bx^{k+1} = \displaystyle\argmax_{\bx \in M} \inner{\bx, \bA\bx^k}$ 
	\item \textbf{if} $\inner{\bx^{k+1}, \bA\bx^{k+1}} = \inner{\bx^k, \bA\bx^k}$ \textbf{then break}
	\item $k = k + 1$
    \end{enumerate}
    \textbf{if } $\inner{\bx^k, \bA\bx^k} \geq \inner{\bx^0, \bA\bx^0}$ \textbf{then return } $\bx^k$
    \textbf{else return } $\bx^0$
\end{algorithm}
This scheme can be seen as optimization of $\inner{\bx, \bA\by}$ over $\bx$ interleaved with that over $\by$.
Thus, it can be seen as a simplified version of Algorithm \ref{algo:quadratic} made for a quadratic case.
Tian \etal ~\cite{Tian2012} proved that this scheme converges to either a circular solution or a fixed point under particular conditions.
Besides, they showed how it can be further improved by replacing the projection onto discrete domain at each step
by a ``soft'' counterpart in continuous domain by using softmax \cite{Gold1996}.

\clearpage
\newpage
