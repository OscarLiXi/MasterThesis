\chapter{A Brief Introduction to Nonlinear Optimization-based Visual-Inertial Odometry }
\label{chap:current}

In the previous chapters, we introduce our loosely-coupled visual-inertial odometry using ESKF, and show our results are slightly better than single visual odometry. Generally, an ESKF update can usually be regarded as one iteration during optimization. Though we apply the bundle adjustment to further improve the estimation quality, the internal correlations among different sensors have been ignored. We hereby explore a more general approach, \eg, a nonlinear optimization way, to solve visual-inertial odometry problem in this chapter. Our presentations in this chapter are based on \cite{leutenegger2015keyframe} and \cite{forster2015imu}. Note that we do not perform any experiments due to the time limitation of this master thesis, the experimental results might be included in future publications.

\section{Cost Function of Visual-Inertial Odometry}
\label{sec:map}

Intuitively, in order to formulate our problem in non-linear optimization way, a cost function to minimize the visual error and IMU temporal error is required. This cost function is precisely the reprojection error and IMU error state we have introduced in Chapter \ref{chap:sensor_fusing}.

We first define the set of true states up to the frame $n$ as
\begin{equation}
	\mX_n \triangleq \{ \vec{x}_t^i \}_{i \in \mK_n},
\end{equation}
where $\{ \vec{x}_t^i \}$ denotes the $i^{th}$ frame of true states, and $\mK_n$ represents the set of keyframes until frame $n$. Therefore our optimal set of true state $\mX_n^\star$ is
\begin{equation}
	\mX_n^\star \triangleq \argmin_{\mX_n} \norm{e_v}^2 + \norm{e_{imu}}^2,
\end{equation}
where $e_v$, $e_{imu}$ are reprojection error, IMU temporal error until the frame $n$ respectively.

Reprojection error can written $e_v$ by Equation (\ref{c1}) and Equation ({\ref{c2}}) as
\begin{equation}
	\norm{e_v}^2 = \sum_{i=1}^n \sum_{j=1}^m \norm{CamProj(\vec{c}_i, \vec{p}_j) - \mI_{ij}}^2, 
\end{equation}
where $n$, $m$ is the number of camera poses and visible 3D points respectively, and $\mI_{ij}$ is the predicted 2D point position corresponding to $j^{th}$ visible 3D point in $i^{th}$ image keyframe.

For IMU temporal error, $\norm{e_{imu}}^2$ can then be denoted via our error state $\vec{x}_e$ as
\begin{equation}
	\norm{e_{imu}}^2 = \sum_{i=1}^n \norm{\vec{x}_e^i}^2,
\end{equation}
where $\{ \vec{x}_e^i \}$ denotes the $i^{th}$ keyframe of error state. Under the assumption of zero-mean Gaussian noise, the cost function can further be written as
\begin{equation}
	\mX_n^\star \triangleq \argmin_{\mX_n} = \sum_{i=1}^n \sum_{j=1}^m \norm{CamProj(\vec{c}_i, \vec{p}_j) - \mI_{ij}}^2_{\mSigma_v} + \sum_{i=1}^n \norm{\vec{x}_e^i}^2_{\mSigma_i},
\end{equation}
where ${\mSigma_v}$ and ${\mSigma_i}$ are covariance matrices for reprojection error and IMU error.

\section{Manifold Nonlinear Optimization}
\label{sec:exponential}

Given the cost function, it is non-trivial to solve it using general nonlinear optimization method (\eg, Gauss-Newton method) since rotational part of $\mX_n$ belongs to a manifold. In this section, we introduce a standard approach for manifold optimization based on \cite{absil2007trust}. 

We first introduce the \textit{exponential map}. A exponential map $exp(\cdot)$ of error state quaternion $\vec{q}_e$, which is very similar to Equation (\ref{q19}), is defined as
\begin{equation}
	\vec{q}_e \triangleq exp(\vec{v}) = \mathrm{e}^{\vec{v}/2} = \begin{bmatrix}
											\cos{\phi / 2}  \\ \vec{u}\sin{\phi / 2} 
									  \end{bmatrix},
\end{equation}
where $\vec{v}$ is the corresponding minimal axis-angle perturbation $\vec{v} = \phi\vec{u}$. Ar exponential map to transform between axis-angle perturbation and rotation matrix can be found in Equation (\ref{q16}). A exponential map maps minimal coordinates (\eg, angular error) into tangent space, which behaves as an Euclidean space locally. Due to this property, we can use the angular error in our error state to form a minimal axis-angle perturbation during the optimization process, and transform it back to quaternion afterwards using exponential map.

In summary, in each optimization iteration, we first perform general optimization approach on such minimal axis-angle perturbation since it behaves as Euclidean space locally. Then the current rotational guess is obtained via exponential map. By using some properties of exponential map, one can decrease the optimization time significantly \cite{forster2015imu}.

\section{Disscussion and Conclusion}
\label{sec:conclusion}

One problem of the optimization approach is that the optimized result is not easily calculated within a short time. Historical information has to be kept in order to apply a global optimization; new keyframes, landmarks and error states will be inserted to cost function continuously. Therefore solving such a cost function usually spends more time than our suggested ESKF framework. Though in \cite{mourikis2007multi}, they solve it by only keep some of previous keyframes, but this is sub-optimal. However, in tightly-coupled visual-inertial SLAM, recent researches has partially addressed complexity problem by proposing \textit{structureless approach} \cite{mourikis2007multi, forster2015imu} to efficiently eliminate all the landmarks (3D points) in optimization step. Also an IMU pre-integration technique \cite{forster2015imu} has been exploited to avoid recomputing relative pose transformations (via IMU measurements) between successive keyframes. These methods have decreased the optimization time significantly, thus non-linear optimization becomes feasible in the real-time VIO system. We aim to explore these approaches since the source codes of \cite{forster2015imu} and a nicely-structured visual-inertial dataset \cite{Burri25012016} have been released recently, unfortunately we are not able to finish coding and debugging until the writing of this thesis.

In this chapter, we introduce an approach to form a optimization cost function via similar notations in this master thesis. To conclude, though our suggested ESKF VIO runs in a constant time complexity, and obtain an acceptable estimation results, we could further improve our approach by focusing on internal connections between visual sensors and IMU sensors. Theoretically, if we can address the high computational complexity well, a more general approach, such as nonlinear optimization, might obtain more accurate results as similar time consumings as our suggested VIO.  
