%% This is an example first chapter.  You should put chapter/appendix that you
%% write into a separate file, and add a line \include{yourfilename} to
%% main.tex, where `yourfilename.tex' is the name of the chapter/appendix file.
%% You can process specific files by typing their names in at the 
%% \files=
%% prompt when you run the file main.tex through LaTeX.
\chapter{Background on Quaternion Algebra}
\label{chap:background}

One important task for this master thesis is to integrate IMU data over time to estimate camera pose (\eg, position and orientation). By assuming movement model, it is straightforward to integrate position in Cartesian space, however integrating orientation in manifold space is often not a easy task. In this chapter, we introduce the quaternion algebra, and explore the way to operate quaternion over time.

\section{Definition of Quaternion}
\label{sec:def_of_quat}

A quaternion $Q$ is defined as,
\begin{equation}\label{q1}
	Q = q_w + q_xi + q_yj + q_zk
\end{equation}
where $\{q_w,q_x,q_y,q_z\} \in \mathbb{R}$, and $\{i,j,k\}$ are three imaginary unit length, \eg, $i^2=j^2=k^2=ijk=-1$.

In most cases, we represent quaternion $Q$ as a four-element vector $\textbf{q}$ composed by above four real number $\{q_w,q_x,q_y,q_z\}$, \ie,
\begin{equation}\label{q2}
	\mathbf{q} \triangleq \left[q_w \ \mathbf{q}_v\right]^T = \left[q_w \ q_x \ q_y \ q_z \right]^T
\end{equation}
where $q_w$ is the real part of $\textbf{q}$, and $q_v$ is a 3-vector to represent imaginary part of $\textbf{q}$. 

It is worth to be noted that there are two different conventions of quaternion $\textbf{q}$, \textit{Hamilton way}~\cite{hamilton1844ii} and \textit{JPL way}~\cite{breckenridge1999quaternions}. In Hamilton convention, the real part $q_w$ is the first component of $\textbf{q}$, \ie, $\left[q_w \ \mathbf{q_v}\right]$, whereas in JPL way, the real part is the fourth component, \ie, $\left[\mathbf{q_v} \ q_w\right]$. To avoid confusions, and considering Hamilton way is more common to use, especially for implementation~\cite{guennebaud2010eigen, hess2007essential}, we hereby claim that we use \textbf{\textit{Hamilton way}} to represent quaternion $\textbf{q}$ throughout this master thesis. 

\section{Properties of Quaternion}
\label{sec:prop_of_quat}

In this section, we will introduce properties of quaternion. 

\textbf{\textit{Summation}} We start with summation of two quaternions \textbf{q} and \textbf{p},
\begin{equation}\label{q7}
\vec{q} + \vec{p} = \left[q_w+p_w \ \mathbf{q}_v+\mathbf{p}_v\right]^T = \left[q_w+p_w \ q_x+p_x \ q_y+p_y \ q_z+p_z \ \right]^T
\end{equation}

\textbf{\textit{Product}} We use $\otimes$ to denote the product operation on quaternions, which gives,
\begin{equation}\label{q3}
	\vec{q} \otimes \vec{p} = \begin{bmatrix}
							  	p_wq_w-p_xq_x-p_yq_y-p_zq_z \\
							  	p_wq_x+p_xq_w+p_yq_z-p_zq_y \\
							  	p_wq_y-p_xq_q+p_yq_w+p_zq_x \\
							  	p_wq_z-p_xq_y-p_yq_z+p_zq_w 
							  \end{bmatrix}
\end{equation}

The product of two quaternions can be expressed as two equivalent matrix products,
\begin{equation}\label{q4}
	\vec{q_1} \otimes \vec{q_2} = Q_1^+\vec{q_2}
\end{equation}
with
\begin{equation}\label{q5}
	Q_1^+ = q_w\mathbb{I} + \begin{bmatrix}
							  	0 &
							  	-\vec{q_v}^T\\
							  	\vec{q_v} &
							  	[\vec{q_v}]_\times
							  \end{bmatrix}
\end{equation}
where $[\vec{q}_v]_\times$ represents the cross-product matrix of $\vec{q}_v$, which the cross-product matrix of a vector $\vec{v}$ is defined by
\begin{equation} \label{q17}
	[\vec{v}]_\times =  \begin{bmatrix}
							0 & -v_z & v_y \\
							v_z & 0 & -v_x \\
							-v_y & v_x & 0 
						\end{bmatrix}
\end{equation}
note that the quaternion product is not commutative, \ie, 
\begin{equation}
	\vec{p} \otimes \vec{q} \neq \vec{q} \otimes \vec{p}
\end{equation}
however it is associative, and distributive over sum, \ie,
\begin{equation} \label{q26}
	\vec{p} \otimes (\vec{q} \otimes \vec{k}) = (\vec{p} \otimes \vec{q}) \otimes \vec{k}
\end{equation}
\begin{equation} \label{q27}
	\vec{p} \otimes (\vec{q} + \vec{k}) = \vec{p} \otimes \vec{q} + \vec{p} \otimes \vec{k}
\end{equation}
\begin{equation} \label{q28}
	(\vec{q} + \vec{k}) \otimes \vec{p} = \vec{q} \otimes \vec{p} + \vec{k} \otimes \vec{p}
\end{equation}

\textbf{\textit{Conjugate}} The conjugate $\vec{q}^*$of a quaternion is defined by
\begin{equation}\label{q6}
	\vec{q}^* \triangleq q_w - \vec{q_v} = \left[ q_w \ - \vec{q_v} \right]^T
\end{equation}

\textbf{\textit{Identity}} We call a quaternion \textbf{q} identical if, for any given quaternion \textbf{p}, $\vec{q} \otimes \vec{p} = \vec{p} \otimes \vec{q} = \vec{p}$. An identical quaternion \textbf{q} satisfies that,
\begin{equation}\label{q8}
	\vec{q} = \left[1 \ 0 \ 0 \ 0 \right]^T
\end{equation}
In this master thesis, we denote identical quaternion as $\vec{q}_{\mathbb{I}}$.

\textbf{\textit{Norm}} The norm of a quaternion $\norm{\vec{q}}$ is defined similar to the norm of a vector, which is,
\begin{equation}\label{q9}
	\norm{\vec{q}} = \sqrt{q_w^2+q_x^2+q_y^2+q_z^2}
\end{equation}

\textbf{\textit{Inverse}} The inverse of a quaternion $\vec{q}^{-1}$ is defined as,
\begin{equation}\label{q10}
	\vec{q}^{-1} = \vec{q}^* / \norm{\vec{q}}
\end{equation}
which leads to,
\begin{equation}\label{q11}
	\vec{q}^* \otimes \vec{q}^{-1} = \vec{q}^{-1} \otimes \vec{q}^*  = \vec{q}_{\mathbb{I}}
\end{equation}

\textbf{\textit{Unit quaternion}} The norm of a unit quaternion $\norm{\vec{q}}$ is $1$, and the inverse of such a unit quaternion is equal to the conjugate of this quaternion,
\begin{equation}\label{q12}
	\vec{q}^{-1} = \vec{q}^*
\end{equation}

\textbf{\textit{Pure quaternion}} A pure quaternion $\vec{q}$ is defined as,
\begin{equation}\label{q13}
	\vec{q} = \left[0 \ \vec{q}_v \right]^T = \left[0 \ q_x \ q_y \ q_z \right]^T
\end{equation}

Let pure quaternion $\vec{q} = \theta\vec{u}$, where $\theta = \norm{\vec{q}}$, we can compute the exponential of $\vec{q}$ with the help of Euler formula,
\begin{equation}\label{q14}
	\mathrm{e}^{\vec{q}} = \mathrm{e}^{\theta\vec{u}} = \cos{\theta} + \vec{u}\sin{\theta} = \left[\cos{\theta} \ \vec{u}\sin{\theta} \right]^T
\end{equation}
which is still a quaternion, and moreover $\mathrm{e}^{\vec{q}}$ is a unit quaternion because its norm $\norm{\mathrm{e}^{\vec{q}}}^2 = \cos{\theta}^2 + \sin{\theta}^2 = 1$.

\section{Quaternions and Rotation operations}
\label{sec:quat_and_rot}

We discuss the relationship between quaternions and rotation operations by first introducing rotation vector $\vec{v}$. 

Given a rotation vector $\vec{v} = \phi\vec{u}$, where $\phi$ is the norm of $\vec{v}$ and $\vec{u}$ is a unit vector, we can rotate a vector $\vec{x}$ by an angle $\phi$ around the axis $\vec{u}$ following right-handed rule, and obtain a new vector $\vec{x^{\prime}}$,
\begin{equation}\label{q15}
	\vec{x^{\prime}} = \vec{x}_{||} + \vec{x}_{\bot}\cos{\phi} + (\vec{u} \times \vec{x})\sin{\phi}
\end{equation}
where $\vec{x}_{||} = (\vec{x} \cdot \vec{u})\vec{u}$ is the component parallel to $\vec{x}$, and $\vec{x}_{\bot} = -\vec{u} \times (\vec{u} \times \vec{x})$ is the component perpendicilar to $\vec{x}$, therefore $\vec{x} = \vec{x}_{||} + \vec{x}_{\bot}$. This formula is known as \textit{vector rotation formular} or \textit{Rodrigues formula}.

We then can define a rotation matrix $\mR$ by rotation vector $\vec{v} = \phi\vec{u}$ by the help of Equation (\ref{q17}) as,
\begin{equation}\label{q16}
	\mR = \mathrm{e}^{[\vec{v}]_\times}
\end{equation}
we can rotate a vector $\vec{x}$ by an angle $\phi$ around the axis $\vec{u}$ using $\mR$ in a clean way,
\begin{equation}\label{q18}
	\vec{x^{\prime}} = \mR\vec{x}
\end{equation}
one can show that result in Equation (\ref{q18}) is equivalent with the result in Equation (\ref{q15}) \cite{wiki:rotationformula}. 

Constructing a unit quaternion $\vec{q}$ by Equation (\ref{q14}) with a rotation vector $\vec{v} = \phi\vec{u}$, 
\begin{equation}\label{q19}
	\vec{q} = \mathrm{e}^{\vec{v}/2} = \begin{bmatrix}
											\cos{\phi / 2}  \\ \vec{v}\sin{\phi / 2} 
									  \end{bmatrix}
\end{equation}
we can rotate a vector $\vec{x}$ by an angle $\phi$ around the axis $\vec{u}$ by,
\begin{equation}\label{q20}
	\vec{x^{\prime}} = \vec{q} \otimes \vec{x} \otimes \vec{q}^{*}
\end{equation}
we then show $\vec{x^{\prime}}$ in Equation (\ref{q20}) is equivalent with $\vec{x^{\prime}}$ in Equation (\ref{q15}).

We transferred the vector $\vec{x}$ into pure quaternion form as,
\begin{equation}\label{q21}
	\vec{x}_q = \left[0 \ \vec{x} \right]^T
\end{equation}
then we rewrite formula (\ref{q20}) as,
\begin{equation}\label{q22}
\begin{bmatrix}
0  \\ \vec{x^{\prime}} 
\end{bmatrix}
=
\begin{bmatrix}
\cos{\phi / 2}  \\ \vec{v}\sin{\phi / 2} 
\end{bmatrix}
\otimes
\begin{bmatrix}
0  \\ \vec{x} 
\end{bmatrix}
\otimes
\begin{bmatrix}
\cos{\phi / 2}  \\ -\vec{v}\sin{\phi / 2}  
\end{bmatrix}
\end{equation}
expanding it by Equation (\ref{q3}), it is easily to show that,
\begin{equation}\label{q23}
	\vec{x^{\prime}} = \vec{x}_{||} + \vec{x}_{\bot}\cos{\phi} + (\vec{u} \times \vec{x})\sin{\phi}
\end{equation}
which is exactly Equation (\ref{q15}).

To summarize here, we can construct a quaternion $\vec{q}$ or a rotation matrix $\mR$ by any rotation vector $\vec{v} = \phi\vec{u}$, we denote such a quaternion as $\vec{q} \{ \vec{v} \}$ and rotation matrix as $\mR \{ \vec{v} \}$ respectively. And a rotation operation of a vector $\vec{x}$ related to $\vec{v}$ can either be expressed as quaternion $\vec{q} \{ \vec{v} \} \otimes \vec{x} \otimes \vec{q} \{ \vec{v} \}^{*}$, or a rotation matrix $\mR \{ \vec{v} \}\vec{x}$. Note that we sometimes simplify $\mR \{ \vec{v} \}$ to $\mR$, and/or $\vec{q} \{ \vec{v} \}$ to $\vec{q}$ in this master thesis.

We also give conversion from rotation matrix $\mR$ to quaternion $\vec{q}$. Knowing that,
\begin{equation}\label{q24}
	\vec{q} \otimes \vec{x} \otimes \vec{q}^{*} = \mR\vec{x}
\end{equation}
we can construct $\mR = \mR \{ \vec{q} \}$ by,
\begin{equation}\label{q25}
\mR = \begin{bmatrix}
		q_w^2+q_x^2-q_y^2-q_z^2 & 2(q_xq_y-q_wq_z) & 2(q_xq_z+q_wq_y) \\
		2(q_xq_y+q_wq_z) & q_w^2-q_x^2+q_y^2-q_z^2 & 2(q_yq_z-q_wq_x)\\
		2(q_xq_z-q_wq_y) & 2(q_yq_z+q_wq_x) & q_w^2-q_x^2-q_y^2+q_z^2 \\
	  \end{bmatrix}
\end{equation} 

and a conversion from quaternion to rotation matrix can be found in \cite{van2005quaternion}.

\section{Time-derivatives on Quaternion}
\label{sec:timed_on_quat}

We introduce the time-derivative on quaternion by first define,
\begin{equation}\label{q29}
	\vec{q}(t+\Delta t) \triangleq \vec{q}(t) \otimes \Delta\vec{q}
\end{equation}
where $\vec{q}(t)$ is the quaternion at time $t$ and $\Delta\vec{q}$ is quaternion transformation within a small period time $\Delta t$. 

One can expand $\Delta\vec{q}$ by Taylor expansions with Equation (\ref{q19}) to,
\begin{equation}\label{q30}
	\Delta\vec{q} = \begin{bmatrix} 1 \\ \dfrac{1}{2}\Delta{\vec{\theta}} \end{bmatrix} + O(\norm{\Delta{\vec{\theta}}}^2)
\end{equation}
where $\Delta{\vec{\theta}}$ is a angular vector corresponding to $\Delta\vec{q}$. In fact, the angular rate $\vec{\omega}$ at time $t$ is defined as,
\begin{equation}\label{q31}
	\vec{\omega} (t) \triangleq \lim_{\Delta{t} \rightarrow 0}\dfrac{\Delta{\vec{\theta}}}{\Delta{t}}
\end{equation}
which is one of measurements we can obtain from IMU sensor.

By definition of the derivative, we can obtain the time-derivative $\dot{\vec{q}}$ of quaternion $\vec{q}$ as,
\begin{equation} \label{q32} 
	\dot{\vec{q}} = \dfrac{d\vec{q}(t)}{dt} \triangleq \lim_{\Delta{t} \rightarrow 0} \dfrac{\vec{q}(t+\Delta t) - \vec{q}(t)}{\Delta{t}}
\end{equation}
which follows,
\begin{equation} \label{q33} 
\begin{split}
	\dot{\vec{q}} \triangleq & \lim_{\Delta{t} \rightarrow 0} \dfrac{\vec{q}(t+\Delta t) - \vec{q}(t)}{\Delta{t}} \\
	=& \lim_{\Delta{t} \rightarrow 0} \dfrac{\vec{q} \otimes \Delta\vec{q} - \vec{q}}{\Delta{t}} \\
	=& \lim_{\Delta{t} \rightarrow 0} \dfrac{\vec{q} \otimes (\begin{bmatrix} 1 \\ \dfrac{1}{2}\Delta{\vec{\theta}} \end{bmatrix} - \begin{bmatrix} 1 \\ 0 \end{bmatrix})}{\Delta{t}} \\
	=& \frac{1}{2}\vec{q} \otimes \begin{bmatrix} 0 \\ \vec{\omega} \end{bmatrix} 
\end{split}
\end{equation}
here we simplify $\vec{q}(t)$ to $\vec{q}$. Then we can obtain the time-derivative on quaternion by writing angular rate into pure quaternion form (\ref{q13}), which is,
\begin{equation} \label{q34} 
	\dot{\vec{q}} = \frac{1}{2}\vec{q} \otimes \vec{\omega}
\end{equation}				  

\section{Time-integration on Quaternion}
\label{sec:timei_on_quat}

To integrate quaternion over time, we explore the relationship between $\vec{q}(t_n)$ and $\vec{q}(t_{n+1})$ where $t_n = n\Delta{t}$. Expanding $\vec{q}(t_{n+1})$ using Taylor series, we have
\begin{equation} \label{q35} 
	\vec{q}(t_{n+1}) = \vec{q}(t_n) + \dot{\vec{q}}(t_n)\Delta{t} + \frac{1}{2!}\ddot{\vec{q}}(t_n)\Delta{t}^2 + \frac{1}{3!}\dddot{\vec{q}}(t_n)\Delta{t}^3+\dots
\end{equation}

Assume that the second order derivative of rotational rate is zero, which is $\ddot{\vec{\omega}} = 0$, we have
\begin{equation} \label{q36} 
	\dot{\vec{q}}(t_{n+1}) = \frac{1}{2}\vec{q}(t_{n}) \otimes \vec{\omega()}(t_{n})
\end{equation}
\begin{equation} \label{q37} 
	\ddot{\vec{q}}(t_{n+1}) = \frac{1}{2^2}\vec{q}(t_{n}) \otimes \vec{\omega}^2(t_{n})+\frac{1}{2}\vec{q}(t_{n}) \otimes \dot{\vec{\omega}}
\end{equation}
\begin{equation} \label{q38} 
	\dddot{\vec{q}}(t_{n+1}) = \frac{1}{2^3}\vec{q}(t_{n}) \otimes \vec{\omega}^3(t_{n}) + \frac{1}{4}\vec{q}(t_{n}) \otimes \dot{\vec{\omega}}\vec{\omega}(t_{n}) + \frac{1}{2}\vec{q}(t_{n}) \otimes \vec{\omega}(t_{n})\dot{\vec{\omega}}
\end{equation}
and so forth and so on. We then get the result of time integration by taking Equation (\ref{q36}, \ref{q37}, \ref{q38}) back into Equation (\ref{q35}).

We hereby gives a stronger assumption that angular rate $\vec{\omega}(t_{n})$ remains constant during a small time period $\Delta{t}$, which is $\dot{\vec{\omega}} = 0$. However, considering the sampling rate of IMU sensor is usually high (> 100 Hz), this assumption actually is general and also gives us a cleaner expression of time integration. 

Given $\dot{\vec{\omega}} = 0$, we can get
\begin{equation} \label{q39} 
	\vec{q}_{n+1} = \vec{q}_n \otimes (1+\frac{1}{2}\vec{\omega}_n\Delta{t}+\frac{1}{2!}(\frac{1}{2}\vec{\omega}_n\Delta{t})^2+\frac{1}{3!}(\frac{1}{2}\vec{\omega}_n\Delta{t})^3+\frac{1}{4!}(\frac{1}{2}\vec{\omega}_n\Delta{t})^4+\dots)
\end{equation}
here we regard the $\vec{q}$ and $\vec{\omega}$ as series, which is exactly
\begin{equation} \label{q40} 
	\vec{q}_{n+1} = \vec{q}_n \otimes \mathrm{e}^{\vec{\omega}\Delta{t}/2}
\end{equation}
we can rewrite it by Equation (\ref{q19}) as,
\begin{equation} \label{q41} 
	\vec{q}_{n+1} = \vec{q}_n \otimes \vec{q}\{ \vec{\omega}_n\Delta{t} \}
\end{equation}
which is called \textbf{\textit{Zeroth order forward integration}} of quaternion over time.

We can obtain \textbf{\textit{Zeroth order backward integration}} by assuming the angular rate remains $\vec{\omega}_{n+1}$ within $\Delta{t}$, then we have
\begin{equation} \label{q42} 
	\vec{q}_{n+1} = \vec{q}_n \otimes \vec{q}\{ \vec{\omega}_{n+1}\Delta{t} \}
\end{equation}
and \textbf{\textit{Zeroth order midward integration}}  by assuming the angular rate holds $\bar{\vec{\omega}}_{n+1} =(\vec{\omega}_{n}+ \vec{\omega}_{n+1})/2$ within $\Delta{t}$,
\begin{equation} \label{q43} 
	\vec{q}_{n+1} = \vec{q}_n \otimes \vec{q}\{ \bar{\vec{\omega}}_{n}\Delta{t} \}
\end{equation}

Though not used in this master thesis, we notice that \cite{trawny2005indirect} gives \textbf{\textit{First order integration}} by assuming angular rate is linear with time, \ie, $\dot{\vec{\omega}} = \frac{\vec{\omega}_{n+1} - \omega_{n}}{\Delta{t}}$, which is
\begin{equation} \label{q44} 
	\vec{q}_{n+1} = \vec{q}_n \otimes \vec{q}\{ \bar{\vec{\omega}}_{n}\Delta{t} \} + \frac{\Delta{t}^2}{48}\vec{q}_n\otimes(\vec{\omega}_n) \otimes \vec{\omega}_{n+1} - \vec{\omega}_{n+1}) \otimes \vec{\omega}_{n}) + \dots
\end{equation}
in our quaternion convention.